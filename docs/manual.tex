\documentclass[12pt]{article}

\usepackage[utf8]{inputenc} 
\usepackage{amsmath}  
\usepackage{graphicx} 
\usepackage{xcolor}
\usepackage{hyperref}
\usepackage[a4paper,margin=1in]{geometry}
\usepackage{listings}

\title{TCI4Keldysh}
\author{Markus Frankenbach}
\date{\today}  

\begin{document}

\maketitle

\begin{section}{Introduction}
\texttt{TCI4Keldysh} can compute imaginary- and real-frequency 4-point vertices from multipoint numerical renormalization group (mpNRG) spectral functions in quantics tensor train (QTT) format.
The vertices can be obtained in their 'full' form or decomposed into a 3d core and lower-dimensional asymptotic contributions.
Further features include the computation of four-point vertices on dense, possibly nonlinear, grids and the computation of 2-4-point correlators, either
in QTT format or on a dense, linear grid.
\end{section}

\begin{section}{Getting started}
The code is exclusively written in Julia. To install all required packages,
simply run \texttt{>> Pkg.instantiate()} in the Julia REPL.
As noted in the \texttt{README.md}, you will need correctly formatted partial spectral functions (.mat files) to use this code.
These are normally provided by the mpNRG code by Lee et.~al.\ \cite{Lee2021}.
Once you have suitable spectral data, change the \texttt{datadir} function in \texttt{src/utils.jl} to
return the parent directory of the directory containing your set of spectral functions.
That's all the setup you need! To run a calculation, either write your own script using
the functionality provided by \texttt{TCI4Keldysh}, or, if you only require a standard feature,
use input files, as explained in the next section.
\end{section}

\begin{section}{Input files}
The code can be run using input files by running:
\begin{verbatim}
    julia scripts/parse_input.jl <path_to_inputfile>
\end{verbatim}
The input file is just a list of key-value pairs, the most important key being
\texttt{jobtype}, as it specifies the type of calculation. For example:
\begin{verbatim}
TCI4Keldysh BEGIN
jobtype matsubarafull
Rrange 0404
tolerance 1.e-2
channel a
flavor_idx 1
PSFpath <path to spectral functions>
TCI4Keldysh END
\end{verbatim}
Available jobtypes and options can be inferred from the \texttt{parse\_input.jl} script.
They will be documented here once users need it.
\end{section}

\begin{section}{Contributing authors}
This document was written by Markus Frankenbach. The \texttt{TCI4Keldysh} code was
written by Markus Frankenbach and Anxiang Ge.
\end{section}

\nocite{*}
\bibliographystyle{plain}
\bibliography{references}

\end{document}