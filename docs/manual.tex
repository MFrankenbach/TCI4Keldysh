\documentclass[12pt]{article}

\usepackage[utf8]{inputenc} 
\usepackage{amsmath}  
\usepackage{graphicx} 
\usepackage{xcolor}
\usepackage{hyperref}
\usepackage{tabularx}
\usepackage{booktabs}
\usepackage[a4paper,margin=2cm]{geometry}
\usepackage{listings}

\title{TCI4Keldysh}
\author{Markus Frankenbach}
\date{\today}  

\begin{document}

\maketitle

\begin{section}{Introduction}
\texttt{TCI4Keldysh} can compute imaginary- and real-frequency 4-point vertices from multipoint numerical renormalization group (mpNRG) spectral functions in quantics tensor train (QTT) format.
The vertices can be obtained in their 'full' form or decomposed into a 3d core and lower-dimensional asymptotic contributions.
Further features include the computation of four-point vertices on dense, possibly nonlinear, grids and the computation of 2-4-point correlators, either
in QTT format or on a dense, linear grid.
\end{section}

\begin{section}{Getting started}
\label{sec:getting_started}
The code is exclusively written in Julia. To install all required packages,
simply run \texttt{>> Pkg.instantiate()} in the Julia REPL.
As noted in the \texttt{README.md}, you will need correctly formatted partial spectral functions (\texttt{.mat} files, converted according to \texttt{README\_data.md}) to use this code.
These can be generated with the mpNRG code by Lee et.~al.\ \cite{Lee2021}.
Off-the-shelf partial spectral functions (PSFs) can be downloaded from OpenDataLMU \href{https://www.google.com/}{here}\textcolor{red}{[TODO:insert link]}.
Once you have suitable spectral data, change the \texttt{datadir} function in \texttt{src/utils.jl} to
return the parent directory of the directory containing your set of spectral functions.
With a single PSF data set, your \texttt{datadir} should be structured as follows:
\begin{verbatim}
    <return value of TCI4Keldysh.datadir()>/
    |
    --<dataset name>/
        |
        --<PSF folder name>/
            |
            --4pt/
                |
                --<4pt PSFs>
            <2/3pt PSFs>
        mpNRG_ph.mat
        mpNRG_pht.mat
        mpNRG_pp.mat
\end{verbatim}
The \texttt{mpNRG\_*.mat} files contain the physical and numerical (broadening etc.) parameters.
Only default broadening parameters are read from these files.
To inform \texttt{TCI4Keldysh} about the temperature of you PSF dataset, modify the function \texttt{TCI4Keldysh.dir\_to\_T()} in \texttt{src/utils.jl}.
That's all you need to run a first calculation! 

To run a calculation, either write your own script using
the functions provided by \texttt{TCI4Keldysh}, or, if you only require a standard feature,
use input files, as explained in the next section.
\end{section}

\begin{section}{Input files}
The code can be run using input files by running:
\begin{verbatim}
    julia --project scripts/parse_input.jl <path_to_inputfile>
\end{verbatim}
The input file is just a list of key-value pairs, the most important key being
\texttt{jobtype}, as it specifies the type of calculation. For example:
\begin{verbatim}
TCI4Keldysh BEGIN
jobtype matsubarafull
Rrange 0404
tolerance 1.e-2
channel a
flavor_idx 1
PSFpath <absolute path to spectral functions>
TCI4Keldysh END
\end{verbatim}
The \texttt{PSFpath} corresponds to the \texttt{<PSF folder name>} in the example given in Sec.\ \ref{sec:getting_started}.
The results of your calculation will by default be written into a directory \texttt{pwtcidir}. To change that directory (\textit{recommended}),
export the environment variable \texttt{PWTCIDIR} before running the calculation.

The following subsections illustrate which jobtypes \texttt{TCI4Keldysh} offers using sample input files.
The most important settings are listed in Tab. \ref{tab:options}.
\begin{table}[h]
    \centering
    \begin{tabular}{c|p{6cm}|p{5cm}}
        \toprule
        name & description & examples\\
        \midrule
            \texttt{jobtype} & The type of calculation to run. & \texttt{matsubarafull} \texttt{matsubaracore} \texttt{keldyshfull}\vfil \texttt{keldyshcore}\\
        \midrule
        \texttt{Rrange} &
        Range \texttt{<fromto>} of how many quantics bits to use in each dimension.
        Calulations will be run for $R\in\{\texttt{from},\ldots,\texttt{to}\}$.
        \textbf{Currently mandatory} even for jobs that don't use it. & \texttt{0710}$\rightarrow R\in\{7,\ldots,10\}$ \\
        \midrule
        \texttt{tolerance} & TCI tolerance. Only relevant in jobs that use TCI, but \textbf{currently mandatory}. & \texttt{1.e-3}\\
        \midrule
        \texttt{channel} & Select channel parametrization. The parametrizations can be seen in the function \texttt{channel\_trafo()} in \texttt{src/utils.jl}. & \texttt{a}\vfil \texttt{p}\vfil \texttt{t}\\
        \midrule
        \texttt{PSFpath} & Absolute path to partial spectral function data. & \texttt{/home/SIAM\_u=0.50/myPSFs}\\
        \midrule
        \texttt{flavor\_idx} & Select either 1 (all spins up) or 2 (two spins up, two down). & \texttt{1}\vfil \texttt{2}\\
        \bottomrule
    \end{tabular}
    \caption{Selected input file options.}
    \label{tab:options}
\end{table}
% More details on available options will be documented here once users need it.
\subsection{QTCI-Compression of Matsubara vertex}
The following input file compresses a full matsubara vertex for 7,8 and 9 quantics bits in each frequency.
To compress only the core, use the \texttt{jobtype} \texttt{matsubaracore}.
\begin{verbatim}
TCI4Keldysh BEGIN
jobtype matsubarafull
Rrange 0709
tolerance 1.e-4
channel p
flavor_idx 1
PSFpath /home/data/SIAM_u=0.50/PSF_nz=4_conn_zavg
TCI4Keldysh END 
\end{verbatim}
\subsection{QTCI-Compression of Keldysh vertex}
\label{subsec:KeldyshCompression}
The following input file compresses a full Keldysh vertex for 12 quantics bits in each frequency.
To compress only the core, use the \texttt{jobtype} \texttt{keldyshcore}.
\begin{verbatim}
TCI4Keldysh BEGIN
jobtype keldyshfull
Rrange 1212
tolerance 1.e-2
channel p
flavor_idx 1
PSFpath /home/data/siam05_U0.05_T0.0005_Delta0.0318
iK 1
unfoldingscheme fused
ommax 0.65
KEV MultipoleKFCEvaluator
coreEvaluator_kwargs cutoff Float64 1.e-6 nlevel Int 4
TCI4Keldysh END 
\end{verbatim}
Keldysh computations come with a few more options than Matsubara: The Keldysh component is
specified with $\texttt{iK}\in\{1,\ldots,15\}$ ($\texttt{iK}=16$ vanishes). The coresponding four Keldysh indices
$(k_1k_2k_3k_4)$ are given by the \texttt{iK}th entry of \texttt{Iterators.product(1:2,1:2,1:2,1:2)}.
The variable \texttt{ommax} sets the frequency box to $[-\texttt{ommax},\texttt{ommax}]^3$.
The \texttt{KEV} option specifies the method of pointwise Keldysh vertex evaluation and can be set to
\texttt{MultipoleKFCEvaluator} or \texttt{KFCEvaluator}. The latter is faster, but uses much more memory because,
for each partial correlator, it contracts one kernel with the PSF during preprocessing.
Therefore, the more generic and recommended method is to use \texttt{MultipoleKFCEvaluator}.
With this option, one can provide a cutoff and a number of levels with the keyword \texttt{coreEvaluator\_kwargs}.
A \texttt{cutoff} of $10^{-6}$ is sufficient for tolerance of $\tau\geq10^{-4}$. Cranking up \texttt{nlevel} speeds
up vertex evaluations, but requires more RAM and preprocessing time.
\subsection{Computation of a Matsubara vertex (without TCI)}
This input files requests computation of the full Matsubara vertex on a $(2^7+1)\times2^7\times2^7$ grid.
The first grid is bosonic and therefore needs an odd number of frequency points to be symmetric.
If you only need the core vertex, change the \texttt{jobtype} to \texttt{conv\_matsubaracore}.
\begin{verbatim}
TCI4Keldysh BEGIN
jobtype conv_matsubarafull
Rrange 0707
tolerance 1.e-2
channel a
flavor_idx 1
PSFpath /home/data/SIAM_u=0.50
TCI4Keldysh END  
\end{verbatim}
\textit{To avoid \texttt{.h5} write clashes, you should only request a single grid size at a time!}
\subsection{Computation of a Keldysh vertex on a linear grid (without TCI)}
Below we give an example for computing a Keldysh vertex on a linear (equidistant) grid of size
$257\times256\times256$. The first grid is centered around zero (`bosonic`). To be symmetric, it needs an additional point.
To compute only the core vertex, use the \texttt{jobtype} \texttt{conv\_keldyshcore}.
\begin{verbatim}
TCI4Keldysh BEGIN
jobtype conv_keldyshfull
Rrange 0808
tolerance 1.e-2
channel a
flavor_idx 1
PSFpath /home/data/SIAM_u=0.50
ommax 0.65
TCI4Keldysh END 
\end{verbatim}
The frequency box is specified by \texttt{ommax} as $[-\texttt{ommax},\texttt{ommax}]^3$.
All Keldysh components of the vertex will be computed.
\textit{To avoid \texttt{.h5} write clashes, you should only request a single grid size at a time!}

\subsection{Computation of a Keldysh vertex on a nonlinear grid (without TCI)}
The following input exemplifies how to evaluate the Keldysh vertex on a nonlinear (e.g.\ logarithmic) grid.
Note that there is a sepearate \texttt{jobtype} for equidistant grids and that, for equal grid sizes,
nonlinear grids are much more costly.
\begin{verbatim}
TCI4Keldysh BEGIN
jobtype nonlin_keldyshfull
Rrange 1212
tolerance 1.e-2
channel t
flavor_idx 1
PSFpath /home/data/SIAM_u=0.50
KEV MultipoleKFCEvaluator
coreEvaluator_kwargs cutoff Float64 1.e-6 nlevel Int 4
frequencygrid loggrid20.h5
foreign_channels a pQFT
TCI4Keldysh END 
\end{verbatim}
The \texttt{frequencygrid} provides the name of a \texttt{.h5} file that specifies the
grid. It contains three \texttt{Float64} vectors with keys \texttt{om1},\texttt{om2},\texttt{om3}.
For details on the \texttt{KEV} and \texttt{coreEvaluator\_kwargs} options, see Sec.\ \ref{subsec:KeldyshCompression}.
\end{section}

\begin{section}{Contributing authors}
This document was written by Markus Frankenbach. The \texttt{TCI4Keldysh} code was
written by Markus Frankenbach and Anxiang Ge.
\end{section}

\nocite{*}
\bibliographystyle{plain}
\bibliography{references}

\end{document}